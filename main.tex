
\documentclass[10pt, twocolumn]{article} %definições iniciais do documento
\usepackage[T1]{fontenc}

%------------------------------------------------
\usepackage{mathptmx}
\usepackage{geometry}   %pacote para configurar as margens
\usepackage[utf8]{inputenc} % Pacote decodificação UTF8 dos caracteres, basicamente para os acentos
\usepackage{graphicx} %para as figuras

\usepackage[explicit]{titlesec}
\usepackage{lipsum}

\renewcommand{\thesection}{\Roman{section}} 
\renewcommand{\thesubsection}{\thesection.\arabic{subsection}}


\titleformat{\section}
  {\normalfont}{\thesection}{12pt}{\MakeUppercase{#1}}
\titleformat{\thesubsection}
  {\normalfont}{\thesubsection}{10pt}{\MakeUppercase{#1}}  
  
  
\geometry{a4paper, left=2cm, right=2cm, bottom=2.5cm, top=2.5cm, headsep=1cm, footskip=1.25cm}


\begin{document}


\twocolumn[
\flushleft
\begin{minipage}{3cm}
\includegraphics{Figuras/unifei.jpg}
\vspace{1cm}
\end{minipage}
\begin{minipage}{6cm}
\begin{tabular}{cccc}
 &  & & \large \textsc{Trabalho de Conclusão de Curso - Maio/2023}  \\ 
 &  & & \large \textsc{Universidade Federal de Itajubá} \\
 &  & & \large \textsc{ENGENHARIA DE CONTROLE E AUTOMAÇÃO}  
\end{tabular}

\end{minipage} 

\begin{center}
    \par Nome
\end{center} 
]

\par \textbf{\textit{Resumo - }Este trabalho }
\vspace{0.25cm}
\par \textbf{Palavras-chave: Separadas, por, vírgulas.}

\section{Introdução}
\par  
\par Por vezes, os exemplos
poderiam ser mostrados de outras maneiras, mas o
objetivo é tentar cobrir todas as possíveis alternativas
contidas em um artigo técnico.

\subsection{Parágrago}

\end{document}
